\documentclass{42-ko}



%******************************************************************************%
%                                                                              %
%                                   프롤로그                                   %
%                                                                              %
%******************************************************************************%

\begin{document}
\title{The Norm}
\subtitle{Version 4}

\summary
{
    본 문서는 42에서 적용 가능한 표준(Norm)을 설명합니다.
    프로그래밍 표준은 코드를 작성할 때에 따라야하는 규칙들을 정의합니다.
    Norm은 기본적으로 커먼 코어 내의 모든 C 프로젝트와
    지정된 모든 프로젝트에 적용됩니다.
}

\maketitle

\tableofcontents



%******************************************************************************%
%                                                                              %
%                                     머리말                                   %
%                                                                              %
%******************************************************************************%
\chapter{머리말}

    \texttt{norminette}은 파이썬으로 작성되었으며 오픈 소스입니다. \\
    저장소는 다음 주소에서 확인할 수 있습니다. https://github.com/42School/norminette \\
    풀 리퀘스트, 제안과 이슈를 환영합니다!

    \newpage


%******************************************************************************%
%
%                                   교육적 설명                                 %
%
%******************************************************************************%
    \chapter{Why ?}

    Norm은 많은 교육학적 요구를 충족시키기 위해 신중하게 만들어졌습니다. 아래의
    모든 선택에 대한 이유는 다음과 같습니다.
    \begin{itemize}

    \item 시퀀싱: 코딩은 크고 복잡한 작업을 긴 일련의 기본 명령으로 나누는 것을 의미합니다.
      이 모든 명령은 차례대로 실행될 것입니다. 소프트웨어를 만들기 시작하는 초보자는 프로젝트를 위해
      모든 개별 명령과 정확한 실행 순서를 완전히 이해한 단순하고 명확한 아키텍처가 필요합니다.
      동시에 여러 명령을 수행하는 암호 언어 구문은 분명히 혼란스러우며 코드의 동일한 부분에 혼합된
      여러 작업을 처리하려는 함수는 오류의 원인입니다.\\
      The Norm은 각 조각의 고유한 작업을 명확하게 이해하고 검증할 수 있고, 실행된 모든 명령의 순서가
      의심의 여지가 없는 곳에 간단한 코드 조각을 만들 것을 요구합니다. 이것이 우리가 함수에서
      최대 25줄을 요구하는 이유이고, 또한 \texttt{for}, \texttt{do .. while}, 또는 삼항 연산자를
      금지하는 이유입니다.

    \item 룩 앤드 필: 일반적인 동료 학습 과정과 동료 평가 중에 친구 및 동료와 교환하는 동안
      코드를 해독하는 데 시간을 쓰지 않고 코드 조각의 논리에 대해 직접 이야기하고 싶어 합니다.\\
      The Norm은 많은 곳에서 함수와 변수의 이름 지정, 들여쓰기, 중괄호 규칙, 탭 및 띄어쓰기에
      대한 지침을 제공하는 특정 룩 앤드 필을 사용하도록 요구합니다. 이것은 익숙해 보이는
      다른 사람의 코드를 부드럽게 살펴볼 수 있으며 코드를 이해하기 전에 코드를 읽는 데 시간을 쓸 필요
      없이 바로 요점을 파악할 수 있게 합니다. The Norm은 또한 트레이드마크입니다. 42 커뮤니티의
      일원으로서, 노동시장에 있을 때 다른 42학생이나 졸업생들이 작성한 코드를 알아볼 수 있을 것입니다.

    \item 장기 비전: 이해할 수 있는 코드를 작성하려는 노력은 그것을 유지하는 가장 좋은 방법입니다.
      당신을 포함한 다른 누군가가 버그를 수정하거나 새로운 기능을 추가해야 할 때마다 이전에 올바른
      방식으로 작업을 수행했다면 그것이 무엇을 하는지 알아내려고 귀중한 시간을 낭비할 필요가 없습니다.
      이렇게 하면 시간이 오래 걸린다는 이유만으로 코드 조각들이 더 이상 유지되지 않는 상황을 피할 수 있고,
      시장에서 성공적인 제품을 가지고 있는 것에 대해 말할 때 차이를 만들 수 있습니다. 그렇게 하는 법은
      빨리 배울수록 좋습니다.

    \item 표준: 당신은 어떤 규칙이나 모든 규칙이 임의적이라고 생각할 수도 있지만, 우리는 실제로 무엇을 어떻게
      해야 할지 생각하고 읽었습니다. 우리는 왜 함수가 짧고 한 가지 일만 해야 하는지, 변수 이름이 의미가 있어야
      하는지, 줄 너비가 80열을 넘지 않아야 하는지, 함수가 많은 매개 변수를 사용하지 않아야 하는지, 주석이
      유용해야 하는 이유 등을 구글에서 검색하는 것을 강력히 권장합니다.

    \end{itemize}


\newpage

%******************************************************************************%
%                                                                              %
%                                The Norm                                      %
%                                                                              %
%******************************************************************************%
\chapter{The Norm}


%******************************************************************************%
%                                   명명 규칙                                  %
%******************************************************************************%
    \section{명명}

        \begin{itemize}

            \item 구조체의 이름은 \texttt{s\_} 로 시작해야만 합니다.

            \item typedef의 이름은 \texttt{t\_} 로 시작해야만 합니다.

            \item 공용체(union)의 이름은 \texttt{u\_} 로 시작해야만 합니다.

            \item 열거형(enum)의 이름은 \texttt{e\_} 로 시작해야만 합니다.

            \item 전역 변수의 이름은 \texttt{g\_} 로 시작해야만 합니다.

            \item 파일 및 디렉터리의 이름에는 소문자, 숫자 및
                '\_' (snake\_case)만이 포함될 수 있습니다.

            \item 파일 및 디렉터리의 이름에는 소문자, 숫자 및
                '\_' (snake\_case)만이 포함될 수 있습니다.

            \item 표준 ASCII 코드표에 없는 문자는 금지됩니다.

            \item 변수, 함수 및 기타 식별자는 스네이크 표기법을 사용해야 합니다.
                대문자는 없고 각 단어는 밑줄 문자로 구분됩니다.

            \item 모든 식별자(함수, 매크로, 자료형, 변수 등)는 영어여야만 합니다.

            \item 객체(변수, 함수, 매크로, 자료형, 파일 또는 디렉터리)는
                가능한 가장 명시적이거나 가장 연상되는 이름을 가져야 합니다.

            \item 프로젝트에서 명시적으로 허용하지 않는 한 상수(const) 및 정적(static)이 아닌
                전역 변수를 사용하는 것은 금지되며 Norm 오류로 간주됩니다.

            \item 파일은 컴파일이 가능해야 합니다. 컴파일되지 않는 파일은
                Norm을 통과할 수 없을 것입니다.
        \end{itemize}
\newpage

%******************************************************************************%
%                                     서식                                     %
%******************************************************************************%
    \section{서식}

            \begin{itemize}

                \item 들여쓰기는 네 칸 크기의 탭으로 이루어져야만 합니다.
                  일반적인 공백 네 칸이 아니라 진짜 탭을 말합니다.

                \item 각 함수는 함수 자체의 중괄호를 제외하고
                  최대 25줄이어야만 합니다.

                \item 각 줄은 주석을 포함해 최대 80자의 열 너비를 가집니다.
                  주의: 탭 들여쓰기는 한 열로 계산하지 않으며,
                  탭이 해당되는 공백의 수 만큼으로 계산됩니다.

                \item 각 함수는 줄 바꿈으로 구분해야 합니다.
                  모든 주석과 전처리기 명령은 함수 바로 위에 있을 수 있습니다.
                  줄 바꿈은 이전 함수 다음에 와야 합니다.

                \item 한 줄에 한 명령만이 존재할 수 있습니다.

                \item 빈 줄은 공백이나 탭 없이 비어 있어야 합니다.

                \item 줄은 공백이나 탭으로 끝날 수 없습니다.

                \item 두 개의 연속된 공백을 사용할 수 없습니다.

                \item 각 중괄호나 제어 구조 다음은 줄 바꿈으로 시작되어야 합니다.

                \item 줄의 끝이 아니라면 각 콤마와 세미콜론 뒤에는 반드시 공백 문자가
                  있어야 합니다.

                \item 각 연산자나 피연산자는 오직 하나의 공백으로 구분해야만 합니다.

                \item 자료형 키워드(int, char, float, 등)와 sizeof를 제외하고,
                  각 C 키워드 다음에는 반드시 공백이 있어야 합니다.

                \item 각 변수 선언은 해당 스코프에 대해서 같은 열로 들여쓰기 해야만 합니다.

                \item 포인터와 함께 쓰이는 별표는 반드시 변수 이름에 붙어있어야 합니다.

                \item 변수 선언은 한 줄에 한 개만 가능합니다.

                \item 전역 변수(허용될 때), 정적 변수, 상수를 제외한
                  선언과 초기화는 같은 줄에 있을 수 없습니다.

                \item 선언문은 반드시 함수의 처음에 있어야 합니다.

                \item 함수 내의 변수 선언과 이후 함수 사이에는 반드시 빈 줄이
                  있어야 합니다. 다른 빈 줄은 함수 내에서 허용되지 않습니다.

                \item 다중 대입은 엄격하게 금지됩니다.

                \item 명령문이나 제어 구조 다음에 새 줄을 추가할 수도 있습니다.
                  그러기 위해서는 들여쓰기와 함께 괄호나 대입 연산자를
                  추가해야 합니다. 연산자는 줄의 시작에 있어야만 합니다.

                \item 제어문(if, while..)에는 단 한 줄만 포함하는 경우를 제외하고
                  반드시 중괄호가 존재해야 합니다.

                \item 함수, 선언자, 제어 구조 다음에 오는 중괄호 앞뒤에는 줄 바꿈이 있어야만 합니다.

            \end{itemize}

            \newpage

            일반적인 예시:
            \begin{42ccode}
int             g_global;
typedef struct  s_struct
{
    char    *my_string;
    int     i;
}               t_struct;
struct          s_other_struct;

int     main(void)
{
    int     i;
    char    c;

    return (i);
}
            \end{42ccode}
            \newpage

%******************************************************************************%
%                              Function parameters                             %
%******************************************************************************%
    \section{Functions}

        \begin{itemize}

            \item A function can take 4 named parameters maximum.

            \item A function that doesn't take arguments must be
                explicitly prototyped with the word "void" as the
                argument.

            \item Parameters in functions' prototypes must be named.

            \item Each function must be separated from the next by
                an empty line.

            \item You can't declare more than 5 variables per function.

            \item Return of a function has to be between parenthesis.

            \item Each function must have a single tabulation between its
                return type and its name.

            \begin{42ccode}
int my_func(int arg1, char arg2, char *arg3)
{
    return (my_val);
}

int func2(void)
{
    return ;
}
            \end{42ccode}

        \end{itemize}
        \newpage


%******************************************************************************%
%                        Typedef, struct, enum and union                       %
%******************************************************************************%
    \section{Typedef, struct, enum and union}

        \begin{itemize}

            \item Add a tabulation when declaring a struct, enum or union.

            \item When declaring a variable of type struct, enum or union,
                add a single space in the type.

            \item When declaring a struct, union or enum with a typedef,
                all indentation rules apply.

            \item Typedef name must be preceded by a tab.

            \item You must indent all structures' names on the same column for their scope.

            \item You cannot declare a structure in a .c file.

        \end{itemize}
        \newpage


%******************************************************************************%
%                                   Headers                                    %
%******************************************************************************%
    \section{Headers - a.k.a include files}

        \begin{itemize}

            \item The things allowed in header files are:
                header inclusions (system or not), declarations, defines,
                prototypes and macros.

            \item All includes must be at the beginning of the file.

            \item You cannot include a C file.

            \item Header files must be protected from double inclusions. If the file is
            \texttt{ft\_foo.h}, its bystander macro is \texttt{FT\_FOO\_H}.

            \item Unused header inclusions (.h) are forbidden.

            \item All header inclusions must be justified in a .c file
                as well as in a .h file.

        \end{itemize}

        \begin{42ccode}
#ifndef FT_HEADER_H
# define FT_HEADER_H
# include <stdlib.h>
# include <stdio.h>
# define FOO "bar"

int		g_variable;
struct	s_struct;

#endif
        \end{42ccode}
        \newpage


%******************************************************************************%
%                                 The 42 header                                %
%******************************************************************************%

   \section{The 42 header - a.k.a start a file with style}

        \begin{itemize}

        \item Every .c and .h file must immediately begin with the standard 42 header :
          a multi-line comment with a special format including useful informations. The
          standard header is naturally available on computers in clusters for various
          text editors (emacs : using \texttt{C-c C-h}, vim using \texttt{:Stdheader} or
          \texttt{F1}, etc...).

        \item The 42 header must contain several informations up-to-date, including the
          creator with login and email, the date of creation, the login and date of the
          last update. Each time the file is saved on disk, the information should be
          automatically updated.

        \end{itemize}
        \newpage


%******************************************************************************%
%                           Macros and Pre-processors                          %
%******************************************************************************%
    \section{Macros and Pre-processors}

        \begin{itemize}

            \item Preprocessor constants (or \#define) you create must be used
                only for literal and constant values.
            \item All \#define created to bypass the norm and/or obfuscate
                code are forbidden. This part must be checked by a human.
            \item You can use macros available in standard libraries, only
                if those ones are allowed in the scope of the given project.
            \item Multiline macros are forbidden.
            \item Macro names must be all uppercase.
            \item You must indent characters following \#if, \#ifdef
                or \#ifndef.
            \item Preprocessor instructions are forbidden outside of global scope.

        \end{itemize}
        \newpage


%******************************************************************************%
%                              Forbidden stuff!                                %
%******************************************************************************%
    \section{Forbidden stuff!}

        \begin{itemize}

            \item You're not allowed to use:

                \begin{itemize}

                    \item for
                    \item do...while
                    \item switch
                    \item case
                    \item goto

                \end{itemize}

            \item Ternary operators such as `?'.

            \item VLAs - Variable Length Arrays.

            \item Implicit type in variable declarations

        \end{itemize}
        \begin{42ccode}
    int main(int argc, char **argv)
    {
        int     i;
        char    string[argc]; // This is a VLA

        i = argc > 5 ? 0 : 1 // Ternary
    }
        \end{42ccode}
        \newpage

%******************************************************************************%
%                                   Comments                                   %
%******************************************************************************%
    \section{Comments}

        \begin{itemize}

            \item Comments cannot be inside functions' bodies.
                Comments must be at the end of a line, or on their own line

            \item Your comments must be in English. And they must be
                useful.

            \item  A comment cannot justify the creation of a carryall or bad function.

        \end{itemize}

        \warn{
            A carryall or bad function usually comes with names that are
            not explicit such as f1, f2... for the function and a, b, i,..
            for the declarations.
            A function whose only goal is to avoid the norm, without a unique
            logical purpose, is also considered as a bad function.
            Please remind that it is desirable to have clear and readable functions that achieve a
            clear and simple task each. Avoid any code obfuscation techniques,
            such as the one-liner..
        }
        \newpage


%******************************************************************************%
%                                    Files                                     %
%******************************************************************************%
    \section{Files}

        \begin{itemize}

            \item You cannot include a .c file.

            \item You cannot have more than 5 function-definitions in a .c file.

        \end{itemize}
        \newpage


%******************************************************************************%
%                                   Makefile                                   %
%******************************************************************************%
    \section{Makefile}

            Makefiles aren't checked by the Norm, and must be checked during evaluation by
            the student.
            \begin{itemize}

                \item The \$(NAME), clean, fclean, re and all
                  rules are mandatory.

                \item If the makefile relinks, the project will be considered
                  non-functional.

                \item In the case of a multibinary project, in addition to
                  the above rules, you must have a rule that compiles
                  both binaries as well as a specific rule for each
                  binary compiled.

                  \item In the case of a project that calls a function from a non-system library
                  (e.g.: \texttt{libft}), your makefile must compile
                  this library automatically.

                  \item All source files you need to compile your project must
                    be explicitly named in your Makefile.

            \end{itemize}


\end{document}
%******************************************************************************%
