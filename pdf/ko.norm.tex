\documentclass{42-ko}
\newcommand\qdsh{\texttt{42sh}}



%******************************************************************************%
%                                                                              %
%                               프롤로그                                       %
%                                                                              %
%******************************************************************************%

\begin{document}
\title{The Norm}
\subtitle{Version 3}

\summary
{
    이 문서는 42에서 적용 가능한 표준(Norm)을 설명합니다. 프로그래밍 표준은 
    코드를 작성할 때 따르는 규칙 집합을 정의합니다. Norm은 기본적으로 이너 써클 
    내의 모든 C 프로젝트와 지정된 모든 프로젝트에 적용됩니다.
}

\maketitle

\tableofcontents



%******************************************************************************%
%                                                                              %
%                                 머리말                                       %
%                                                                              %
%******************************************************************************%
\chapter{머리말}

    Norm은 파이썬으로 작성되었으며, 오픈 소스입니다. \\
    이 저장소는 https://github.com/42School/norminette 에서 이용할 수 있습니다.\\
    풀 리퀘스트, 제안 그리고 이슈는 환영합니다!

%******************************************************************************%
%                                                                              %
%                                The Norm                                      %
%                                                                              %
%******************************************************************************%
\chapter{The Norm}


%******************************************************************************%
%                             명명 규칙                                        %
%******************************************************************************%
    \section{명칭}

        \begin{itemize}

            \item 구조체의 이름은
                \texttt{s\_} 로 시작해야 합니다.

            \item typedef된 자료형의 이름은
                \texttt{t\_} 로 시작해야 합니다.

            \item 공용체의 이름은 \texttt{u\_} 로 시작해야 합니다.

            \item enum의 이름은 \texttt{e\_} 로 시작해야 합니다.

            \item 전역변수의 이름은 \texttt{g\_} 로 시작해야 합니다.

            \item 변수와 함수의 이름에는 소문자, 숫자 및 '\_' (Unix Case)만
                포함될 수 있습니다.

            \item 파일 및 디렉터리의 이름에는 소문자, 숫자 및 '\_' (Unix Case)만 
                포함될 수 있습니다.

            \item 표준 아스키코드가 아닌 문자는 사용할 수 없습니다.

            \item 변수, 함수 및 기타 식별자는 스네이크 케이스를 사용해야 합니다.
                대문자는 없고, 각 단어는 밑줄로 구분됩니다.

            \item 모든 식별자들(함수, 매크로, 자료형, 변수 등)은 
                영어로 되어 있어야 합니다.

            \item 객체(변수, 함수, 매크로, 자료형, 파일 또는 디렉터리)는 
                가능한 가장 명시적이거나 가장 연상되는 이름을 가져야 합니다.

            \item 프로젝트에서 명시적으로 허용하지 않는 한, 상수가 아니거나 
                정적이지 않은 전역변수의 사용은 금지되어 있으며 Norm 오류로 간주됩니다.

            \item 파일은 컴파일이 가능해야 합니다. 컴파일이 되지 않는 파일은
                Norm을 통과할 수 없을 것입니다.
        \end{itemize}
\newpage

%******************************************************************************%
%                                 서식                                         %
%******************************************************************************%
    \section{서식}

            \begin{itemize}

                \item 들여쓰기는 네 칸 크기의 탭으로 이루어져야 합니다.
                    일반적인 공백 네 칸이 아니라, 진짜 탭을 말합니다.

                \item 각 함수는 함수 자체의 중괄호를 제외하고 
                    최대 25줄이어야 합니다.

                \item Each line must be at most 80 columns wide, comments
                  included. Warning: a tabulation doesn't count
                  as a column, but as the number of spaces it
                  represents.

                \item Each function must be separated by a newline. Any comment or preprocessor instruction
                    can be right above the function. The newline is after the previous function.

                \item One instruction per line.

                \item An empty line must be empty: no spaces or tabulations.

                \item A line can never end with spaces or tabulations.

                \item You can never have two consecutive spaces.

                \item You need to start a new line after each curly bracket
                  or end of control structure.

                \item Unless it's the end of a line, each comma or semi-colon
                  must be followed by a space.

                \item Each operator or operand must be separated by one
                 - and only one - space.

                \item Each C keyword must be followed by a space, except for
                  keywords for types (such as int, char, float, etc.),
                  as well as sizeof.

                \item Each variable declaration must be indented on the same
                  column for its scope.

                \item The asterisks that go with pointers must be stuck to
                  variable names.

                \item One single variable declaration per line.

                \item Declaration and an initialisation cannot be
                  on the same line, except for global variables (when allowed),
                  static variables, and constants.

                \item Declarations must be at the beginning of a function.

                \item In a function, you must place an empty line between 
                    variable declarations and the remaining of the function.
                    No other empty lines are allowed in a function.

                \item Multiple assignments are strictly forbidden.

                \item You may add a new line after an instruction or
                  control structure, but you'll have to add an
                  indentation with brackets or assignment operator.
                  Operators must be at the beginning of a line.

                \item Control structures (if, while..) must have braces, unless they contain a single 
                    line or a single condition.

            \end{itemize}

            \newpage

            일반적인 예시:
            \begin{42ccode}
int             g_global;
typedef struct  s_struct
{
    char    *my_string;
    int     i;
}               t_struct;
struct          s_other_struct;

int     main(void)
{
    int     i;
    char    c;

    return (i);
}
            \end{42ccode}
            \newpage

%******************************************************************************%
%                              함수 매개변수                                    %
%******************************************************************************%
    \section{함수}

        \begin{itemize}

            \item A function can take 4 named parameters maximum.

            \item A function that doesn't take arguments must be
                explicitly prototyped with the word "void" as the
                argument.

            \item Parameters in functions' prototypes must be named.

            \item Each function must be separated from the next by
                an empty line.

            \item You can't declare more than 5 variables per function.

            \item Return of a function has to be between parenthesis. 

            \item Each function must have a single tabulation between its
                return type and its name.

            \begin{42ccode}
int my_func(int arg1, char arg2, char *arg3)
{
    return (my_val);
}

int func2(void)
{
    return ;
}
            \end{42ccode}

        \end{itemize}
        \newpage


%******************************************************************************%
%              자료형, 구조체(struct), 열거형(enum)과 공용체(union)              %
%******************************************************************************%
    \section{Typedef, struct, enum and union}

        \begin{itemize}

            \item Add a tabulation when declaring a struct, enum or union.

            \item When declaring a variable of type struct, enum or union,
                add a single space in the type.

            \item When declaring a struct, union or enum with a typedef,
                all indentation rules apply. You must align the typedef's name
                with the struct/union/enum's name.

            \item You must indent all structures' names on the same column for their scope.

            \item You cannot declare a structure in a .c file.

        \end{itemize}
        \newpage


%******************************************************************************%
%                                   Headers                                    %
%******************************************************************************%
    \section{Headers}

        \begin{itemize}

            \item The things allowed in header files are:
                header inclusions (system or not), declarations, defines,
                prototypes and macros.

            \item All includes must be at the beginning of the file.

            \item You cannot include a C file.

            \item Header files must be protected from double inclusions. If the file is
            \texttt{ft\_foo.h}, its bystander macro is \texttt{FT\_FOO\_H}.

            \item Unused header inclusions (.h) are forbidden.

            \item All header inclusions must be justified in a .c file
                as well as in a .h file.

        \end{itemize}

        \begin{42ccode}
#ifndef FT_HEADER_H
# define FT_HEADER_H
# include <stdlib.h>
# include <stdio.h>
# define FOO "bar"

int		g_variable;
struct	s_struct;

#endif
        \end{42ccode}
        \newpage

%******************************************************************************%
%                           Macros and Pre-processors                          %
%******************************************************************************%
    \section{Macros and Pre-processors}

        \begin{itemize}

            \item Preprocessor constants (or \#define) you create must be used
                only for literal and constant values.
            \item All \#define created to bypass the norm and/or obfuscate
                code are forbidden. This part must be checked by a human.
            \item You can use macros available in standard libraries, only
                if those ones are allowed in the scope of the given project.
            \item Multiline macros are forbidden.
            \item Macro names must be all uppercase.
            \item You must indent characters following \#if, \#ifdef
                or \#ifndef.

        \end{itemize}
        \newpage


%******************************************************************************%
%                              금지 사항!                                %
%******************************************************************************%
    \section{금지 사항!}

        \begin{itemize}

            \item 다음 구문은 사용이 금지됩니다:

                \begin{itemize}

                    \item for
                    \item do...while
                    \item switch
                    \item case
                    \item goto

                \end{itemize}

            \item Ternary operators such as `?'.

            \item VLAs - Variable Length Arrays.

            \item Implicit type in variable declarations

        \end{itemize}
        \begin{42ccode}
    int main(int argc, char **argv)
    {
        int     i;
        char    string[argc]; // This is a VLA

        i = argc > 5 ? 0 : 1 // Ternary
    }
        \end{42ccode}
        \newpage

%******************************************************************************%
%                                   주석                                       %
%******************************************************************************%
    \section{주석}

        \begin{itemize}

            \item 주석은 함수 내부에 있을 수 없습니다. 
				주석은 줄 끝에 있거나 별개의 줄에 있어야 합니다.

            \item 주석은 영어여야 합니다. 그리고 유용해야 합니다.

            \item 주석은 "쓰레기같은" 함수를 정당화할 수 없습니다.

        \end{itemize}
        \newpage


%******************************************************************************%
%                                    파일                                      %
%******************************************************************************%
    \section{파일}

        \begin{itemize}

            \item .c 파일을 인쿨루드할 수 없습니다.

            \item 하나의 .c 파일에 5개보다 많은 함수 정의를 포함할 수 없습니다.

        \end{itemize}
        \newpage


%******************************************************************************%
%                                   Makefile                                   %
%******************************************************************************%
    \section{Makefile}

            Makefiles aren't checked by the Norm, and must be checked during evaluation by 
            the student.
            \begin{itemize}

                \item The \$(NAME), clean, fclean, re and all
                  rules are mandatory.

                \item If the makefile relinks, the project will be considered
                  non-functional.

                \item In the case of a multibinary project, in addition to
                  the above rules, you must have a rule that compiles
                  both binaries as well as a specific rule for each
                  binary compiled.

                  \item In the case of a project that calls a function from a non-system library
                  (e.g.: \texttt{libft}), your makefile must compile
                  this library automatically.

                  \item All source files you need to compile your project must
                    be explicitly named in your Makefile.

            \end{itemize}



\end{document}
%******************************************************************************%
